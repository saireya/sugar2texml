% --- Beamer用共通Tex設定ファイル ---

% パッケージ読み込み
\usepackage{luatexja, multirow, alltt, wrapfig}
\usepackage[style=numeric,backend=biber]{biblatex}

% フォントをsans-serifに。
\renewcommand{\kanjifamilydefault}{\gtdefault}
% ツールバーを非表示に。
\setbeamertemplate{navigation symbols}{}
% ブロックの影をなくす(と境界が分からなくなる)。
% \setbeamertemplate{blocks}[rounded][shadow=false]
% フレームヘッダの背景、通常ブロックの背景を灰色に。
\definecolor{gray}{cmyk}{.2,.2,.2,.8}
% structureを変更すると、強調色まで灰色になる。
% \setbeamercolor{structure}{fg=gray}
\setbeamercolor{frametitle}{ bg=gray, fg=white}
\setbeamercolor{block title}{bg=gray, fg=white}
% ブロックの背景色を薄める。
\setbeamercolor{block body}{bg=gray!02!white}
\setbeamercolor{block body alerted}{bg=red!02!white}
\setbeamercolor{block body example}{bg=green!02!white}
% 青背景版
% \setbeamercolor{block body}{bg=blue!02!white}

% 箇条書きアイテムマーカーを四角に変更
\setbeamertemplate{items}[circle]

% 表題の余白。
\abovecaptionskip=1pt
\belowcaptionskip=1pt

% 基本情報。
\title{\Title}
\author{\Author}
\date{}

% 共通ファイル読み込み
% --- 共通TeX設定ファイル ---

% 基本情報
\title{\Title}
\author{\Author}

% align環境。
\usepackage{amsmath}

% 数式にOpenTypeフォントを使用。
\usepackage{unicode-math}
\unimathsetup{math-style=ISO,bold-style=ISO}
\setmathfont{XITS Math}

% PDFの各種設定。
\hypersetup{pdfencoding=auto,unicode=true,pdftitle={\Title},pdfauthor={\Author},pdfcreator=,bookmarksopen,bookmarksnumbered}

% MathJaxと共通のマクロを読み込み。
\renewcommand{\rm}[1]{\mathrm{#1}}
% --- 共通TeXマクロファイル ---
% \[

% 集合。
\newcommand{\mbb}[1]{\mathbb{#1}}
\newcommand{\N}{\mbb{N}}
\newcommand{\Z}{\mbb{Z}}
\newcommand{\Q}{\mbb{Q}}
\newcommand{\R}{\mbb{R}}
\renewcommand{\C}{\mbb{C}}
\newcommand{\num}[1]{{{}^\#{#1}}}

% 量化記号。
\newcommand{\A }{{}^\forall}
\newcommand{\E }{{}^\exists}
\newcommand{\EI}{{}^{\exists1}}

% 矢印記号。
\newcommand{\Rarrow }{\to}
\newcommand{\rarrow }{\Longrightarrow}
\newcommand{\Lrarrow}{\longleftrightarrow}
\newcommand{\darrow }{\overset{\mathrm{def}}{\lrarrow}}

% 関数
\newcommand{\tr    }{\rm{tr}}
\newcommand{\sgn   }{\rm{sgn}}
\newcommand{\Arcsin}{\rm{Arcsin}}
\newcommand{\Arccos}{\rm{Arccos}}
\newcommand{\Arctan}{\rm{Arctan}}
\newcommand{\Int}{\int\hspace{-5.5pt}}
\newcommand{\argmax}{\mathop{\rm{arg~max}}\limits}
\newcommand{\argmin}{\mathop{\rm{arg~min}}\limits}

% ベクトル。
\newcommand{\V}[1]{\mathbfit{#1}}
\newcommand{\diag}{\mathop{\oplus}}
\newcommand{\Diag}{\rm{diag}}
\newcommand{\rank}{\rm{rank}}
\newcommand{\Tr}[1]{{}^t#1}

% 場合の数。
\newcommand{\twocmd}[3]{{}_{#1} \mathrm{#3}_{#2}}
\newcommand{\p}[2]{\twocmd{#1}{#2}{P}}
\newcommand{\h}[2]{\twocmd{#1}{#2}{H}}
\renewcommand{\c}[2]{\twocmd{#1}{#2}{C}}

% Boolean
\renewcommand{\and}{\wedge}
\renewcommand{\not}{\hspace{1pt}\overline}
\newcommand{\nor}{\hspace{1pt}\bar{\vee}\hspace{1pt}}
\newcommand{\nand}{\hspace{1pt}\bar{\and}\hspace{1pt}}
\newcommand{\xor}{\oplus}

% 丸囲み数字
\newcommand{\numcirc}[1]{\textcircled{\scriptsize #1}}

% \]


% ルビ。
\newcommand{\ruby}[2]{%
\leavevmode
\setbox0=\hbox{#1}%
\setbox1=\hbox{\tiny #2}%
\ifdim\wd0>\wd1 \dimen0=\wd0 \else \dimen0=\wd1 \fi
\hbox{%
\ltjsetparameter{kanjiskip=0pt plus 2fil, xkanjiskip=0pt plus 2fil}
\vbox{%
\hbox to \dimen0{%
\tiny \hfil#2\hfil}%
\nointerlineskip
\hbox to \dimen0{\mathstrut\hfil#1\hfil}}}}

% 脚注。
\newcommand{\fn}[1]{\footnote{#1}}
\newcommand{\ft}[2][\value{footnote}]{\footnotetext[#1]{#2}}
\newcommand{\fm}{\footnotemark}

% 場合分け。
\newenvironment{casesproof}
{
  \renewcommand{\labelenumi}{[\arabic{enumi}]}
  \begin{enumerate}
}{
  \end{enumerate}
}

% multicols内ではtable,figureを直接使えない。
\makeatletter
\newenvironment{tablehere} {\def\@captype{table}}{}
\newenvironment{figurehere}{\def\@captype{figure}}{}
\makeatother

% csv
\usepackage[table]{xcolor}
\usepackage{alltt, longtable, multicol, csvsimple, colortbl}
\PassOptionsToPackage{table}{xcolor}

% 表に関する色。偶数行の色変更はtabularタグの設定で行っている。
\definecolor{lightpurple}{rgb}{1,0.8,1}
\definecolor{morepurple}{rgb}{0.8,0,0.8}
% 表の線の色を変更。
\arrayrulecolor{morepurple}

% listings
\usepackage{lstlinebgrd}

% コード表示部の設定。
\definecolor{darkblue} {rgb}{0.0,  0.0,  0.6}
\definecolor{deeppink} {rgb}{1.0,  0.07, 0.57}
\definecolor{deepgreen}{rgb}{0.0,  0.5,  0.0}
\definecolor{lightgray}{rgb}{0.96, 0.96, 0.96}
\definecolor{silver}   {rgb}{0.68, 0.68, 0.68}
\lstset{
	frame=single,
	stepnumber=1,
	tabsize=4,
	keepspaces=true,
	numbers=left,
	numbersep=5pt,
	numberstyle =\scriptsize\color{silver}\sffamily,
	basicstyle  =\scriptsize\ttfamily\gtfamily\sffamily,
	stringstyle =\color{blue},
	keywordstyle=\color{deeppink},
	commentstyle=\tiny\color{deepgreen},
	lineskip=-0.3\zw,
	columns=[l]{fullflexible},
	breaklines=true,
	showstringspaces=false,
	linebackgroundcolor={\ifodd\value{lstnumber}\color{lightgray}\fi}
}
\lstalias{log}{}
\lstalias{m}{octave}
\lstalias{pde}{java}
\lstalias{py}{python}
\lstalias{rb}{ruby}
\lstalias{scm}{lisp}
\lstalias{txt}{}
\lstdefinelanguage{XML}{
	morestring=[b]",
	morestring=[s]{>}{<},
	morecomment=[s]{<?}{?>},
	morekeywords={xmlns,version,type},
	identifierstyle=\color{darkblue},
}
\lstdefinelanguage{diff}{
	morecomment=[f][\color{blue}]{@@},     % group identifier
	morecomment=[f][\color{red}]-,         % deleted lines 
	morecomment=[f][\color{blue}]+,        % added lines
	morecomment=[f][\color{magenta}]{---}, % Diff header lines (must appear after +,-)
	morecomment=[f][\color{magenta}]{+++},
}




% Beamerでもemで強調できるように。
\renewcommand{\em}[1]{\structure{#1}}
